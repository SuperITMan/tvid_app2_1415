Code structure by Virginie Van den Schriek

%----------------------------------------------------------------------------------------
%	PACKAGES AND OTHER DOCUMENT CONFIGURATIONS
%----------------------------------------------------------------------------------------
\documentclass[a4paper,10pt,final,fleqn]{article}
\usepackage[frenchb]{babel}
\usepackage{fontenc}
\usepackage{fancyhdr} % Required for custom headers
\usepackage{lastpage} % Required to determine the last page for the footer
\usepackage{extramarks} % Required for headers and footers
\usepackage{graphicx} % Required to insert images
\usepackage[utf8]{inputenc}
\usepackage{apacite}
\usepackage{url}
\usepackage[normalem]{ulem}
\usepackage{enumitem}
\usepackage{mathtools}


% Margins
%\topmargin=-0.45in
\evensidemargin=0in
\oddsidemargin=0in
\textwidth=6in
\textheight=9.0in
\headsep=0.25in 

%\linespread{1.1} % Line spacing

% Set up the header and footer
\pagestyle{fancy}
\lhead{\AuthorName} % Top left header
\chead{\Course} % Top center header
\rhead{\Date} % Top right header
\lfoot{\lastxmark} % Bottom left footer
\cfoot{} % Bottom center footer
\rfoot{\ \thepage\ / 	\pageref{LastPage}} % Bottom right footer
\renewcommand\headrulewidth{0.4pt} % Size of the header rule
\renewcommand\footrulewidth{0.4pt} % Size of the footer rule



%\setlength\parindent{0pt} % Removes all indentation from 
\setcounter{tocdepth}{2}


\newcommand{\Course}{EPHEC} % Course/class
\newcommand{\AuthorName}{Groupe 2.x} % Your name
\newcommand{\Date}

\title{
\parbox{15cm}
{\includegraphics[width=4cm]{ephec.png} \\ 
  Louvain-la-neuve\\
    \vspace{3cm}
	\begin{center}\sf\bfseries\Huge
		\rule{15cm}{1pt}
		\medskip
		Réponses aux questionnaire-guide \\
		\huge APP2 : VoIP(1)\\
		\vspace{.5cm}
		%\Large Enoncé destiné aux 2TL1\\ [-2mm]
		\rule{15cm}{1pt}
	\end{center}
	\vspace{3cm}
}} 

\author{Ramiro-Gonzalez, De Mets, Vergotte, Goerges, Monroe} \date{\today}


\begin{document}
\thispagestyle{empty}

\maketitle

\newpage

\section{Recherche Théorique}
	
	\subsection{Transmissions vocales}
		
		\begin{description}[style=nextline]

			\item[Comment la voix est-elle numérisée? Indiquer les différentes
			étapes. Quels paramètres vont influencer la qualité de la communication?
			Quel débit obtient-on, en fonction de différents cas de figure (téléphonie,
			MP3 vs Hi-Fi, mono vs stéréo, etc )? Expliquer les calculs.]
			Numérisation d'un signal audio : 
				begin{itemize}
					\item La voix constitue un signal \textbf{analogique}.
					\item Ce signal est ensuite échantillonné selon Shannon-Nyquist. Par exemple
					pour un signal de qualité téléphonique, la bande-passante du signal est de 4KHz,
					donc la fréquence d'échantillonnage doit être de 2*4000Hz(Shannon-Nyquist).
					On a donc un taux de 8000 échantillons/seconde.
					\item Ensuite le signal échantillonné est quantifié pour attribuer un code
					binaire à chaque échantillon, typiquement on quantifiera ce signal sur 8 bits,
					ce qui offre \begin{equation} 2^8 = 256 \end{equation} valeurs possibles.
					\item On obtient donc un signal sonore numérisé d'un débit binaire de :
						Taux d'échantillonnage * Codage = \begin{equation} 8000 \ast 8 = 64Kbps \end{equation}.
				\end{itemize}
			Les paramètres qui vont influencer la qualité de la communication au niveau du signal sont la bande-passante
			allouée au dit signal (et donc le taux d'échantillonnage) et le nombre de bits sur lequel les échantillons
			vont être codés. Plus ces valeurs seront élevées plus le signal numérique sera proche de l'original
			analogique, mais plus le signal sera "lourd".
			Le débit est le résultat de l'équation suivante :
			\begin{equation} Débit = \frac{(Taux d'échantillonnage * Nombre de bits * Nombre de voies)}{Type de compression}\end{equation}
			Le taux d'échantillonnage représente la qualité du son (Radio 22.05KHz, CD 44.1KHz, Téléphone 8KHz).
			Le nombre de voies indique le type mono(Une voie), stéréo (Deux voies), etc.
			Un dernier paramètre est le type de compression, tel MP3 qui va réduire de 12 à 200 fois le débit.


			\item[Lors de i'\textbf{échantillonnage}, une contrainte physique doit être prise en compte pour
			éviter des distorsions. Quelle est-elle? Enoncez le principe qui l'explicite. Que risque-t-on si
			elle n'est pas respectée?]
			C'est le théorème, cité précédemment, de Shannon-Nyquist :
				\begin{quotation}
				La représentation discrète d'un signal par des échantillons régulièrement espacés exige une 
				fréquence d'échantillonnage supérieure au double de la fréquence maximale présente dans ce signal.
				\end{quotation}
			Si ce principe n'est pas respecté lors de l'échantillonnage, les copies adjacentes au signal (Transformées 
			de Fourrier) vont se chevaucher et altérer le son produit.

			\item[De la \textbf{compression} peut être utilisée pour réduire la bande passante. Comment réaliser cette
			compression dans le cas des transmissions vocales?]
			Cette compression est réalisée via des \textbf{Codecs} (Codeur/Décodeur), qui sont des programmes de compression qui
			encodent le signal vocal en données digitales compressées pour l'envoi et décompressées au niveau du récepteur.

			\item[Listez quelques standards pour l'\textbf{encodage} de la voix, en mentionnant leurs
			caractéristiques respectives]
			\begin{table}[h]
			\begin{tabular}{lll}
			\textbf{Codec} & \textbf{Bande Passante(Kbps)} & \textbf{Description}                                               \\
			G.711          & 64                            & Offre une transmission précise de la parole.                       \\
			G.722          & 48/56/64                      & Adaptation du taux de compression en cas de congestion réseau.     \\
			G.723.1        & 5.3/6.3                       & Haut taux de compression avec une bonne qualité audio.             \\
			G.726          & 16/24/32/40                   & Version améliorée du G.721                                         \\
			G.729          & 8                             & Bonne utilisation de la bande, tolérant aux erreurs.               \\
			GSM            & 13                            & Haut taux de compression, utilisé dans les téléphones cellulaires. \\
			iLBC           & 15                            & Gratuit, robuste face aux pertes.                                  \\
			Speex          & 2.15/44                       & Minimise l'utilisation de la bande avec un débit variable.        
			\end{tabular}
			\end{table}

			\item[Quels sont les besoins d'une transmission vocale à travers un réseau? Quelle bande-passante va-t-elle occuper(fixe, variable)?
			Est-elle sensible aux pertes? au délai? au jitter? Quel \textbf{protocole de transport} sera le plus adapté à ce type de transmission?]


			\item[Quelle sera la différence entre une transmission de type streaming audio et une transmission de type
			téléphonie sur internet? Quel mécanisme peut-on utiliser dans un cas mais pas dans l'autre pour contrer certaines
			imperfections du réseau IP?]

		\end{description}

	\subsection{Infrastructure réseau et technologie d'accès.}
	
\end{document}
